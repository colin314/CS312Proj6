\documentclass{article}

\usepackage{amsmath,graphicx}
\usepackage{sectsty}
\usepackage[letterpaper, margin=0.5cm]{geometry}
\usepackage{amsmath,graphicx}
\usepackage{multicol}
\usepackage{array}
\usepackage{csvsimple}
\usepackage{booktabs}
\usepackage{multirow}
\usepackage{float}
\usepackage{array}

\setlength{\heavyrulewidth}{1.5pt}
\setlength{\abovetopsep}{4pt}

\newcommand{\centered}[1]{\begin{tabular}{l} #1 \end{tabular}}

\graphicspath{./}

\begin{document}
    \begin{center}
        \begin{table}[!htbp]
            \centering
            \caption{Tour Distance Over Time at Various Iteration Counts}
            \begin{tabular}{cccccc}
                \toprule
                \multicolumn{1}{c|}{
                    \parbox[t]{1.5cm}{Number\\ of cities}} & 
                \multicolumn{5}{c}{Number of Iterations} \\
                \midrule
                
                & 50 & 100 & 250 & 500 & 1000 \\
                
                15 &
                \centered{\includegraphics[width=0.14\textwidth]{sVal100/Annealing_15Pts_50it.png}} &
                \centered{\includegraphics[width=0.14\textwidth]{sVal100/Annealing_15Pts_100it.png}} &
                \centered{\includegraphics[width=0.14\textwidth]{sVal100/Annealing_15Pts_250it.png}} &
                \centered{\includegraphics[width=0.14\textwidth]{sVal100/Annealing_15Pts_500it.png}} &
                \centered{\includegraphics[width=0.14\textwidth]{sVal100/Annealing_15Pts_1000it.png}} \\
                
                30 &
                \centered{\includegraphics[width=0.14\textwidth]{sVal100/Annealing_30Pts_50it.png}} &
                \centered{\includegraphics[width=0.14\textwidth]{sVal100/Annealing_30Pts_100it.png}} & 
                \centered{\includegraphics[width=0.14\textwidth]{sVal100/Annealing_30Pts_250it.png}} & 
                \centered{\includegraphics[width=0.14\textwidth]{sVal100/Annealing_30Pts_500it.png}} & 
                \centered{\includegraphics[width=0.14\textwidth]{sVal100/Annealing_30Pts_1000it.png}} \\
                60 &
                \centered{\includegraphics[width=0.14\textwidth]{sVal100/Annealing_60Pts_50it.png}} &
                \centered{\includegraphics[width=0.14\textwidth]{sVal100/Annealing_60Pts_100it.png}} &
                \centered{\includegraphics[width=0.14\textwidth]{sVal100/Annealing_60Pts_250it.png}} & 
                \centered{\includegraphics[width=0.14\textwidth]{sVal100/Annealing_60Pts_500it.png}} &
                \centered{\includegraphics[width=0.14\textwidth]{sVal100/Annealing_60Pts_1000it.png}} \\
                100 &
                \centered{\includegraphics[width=0.14\textwidth]{sVal100/Annealing_100Pts_50it.png}} &
                \centered{\includegraphics[width=0.14\textwidth]{sVal100/Annealing_100Pts_100it.png}} &
                \centered{\includegraphics[width=0.14\textwidth]{sVal100/Annealing_100Pts_250it.png}} &
                \centered{\includegraphics[width=0.14\textwidth]{sVal100/Annealing_100Pts_500it.png}} \\
                %\centered{\includegraphics[width=0.14\textwidth]{sVal100/Annealing_100Pts_1000it.png}} &
                200 &
                \centered{\includegraphics[width=0.14\textwidth]{sVal100/Annealing_200Pts_50it.png}} &
                \centered{\includegraphics[width=0.14\textwidth]{sVal100/Annealing_200Pts_100it.png}} &
                \centered{\includegraphics[width=0.14\textwidth]{sVal100/Annealing_200Pts_250it.png}} & 
                \centered{\includegraphics[width=0.14\textwidth]{sVal100/Annealing_200Pts_500it.png}} & 
                \centered{\includegraphics[width=0.14\textwidth]{sVal100/Annealing_200Pts_1000it.png}} \\
            \end{tabular}
        \end{table}

        \begin{tabular}{|c||c|c|c|c|c|}
            \hline
            & 1000 & 500 & 200 & 100 & 50 \\
            \hline\hline
            15 & 25.6\% & 1.0\% & 19.0\% & 3.9 \% & 15.2\% \\
            \hline
            30 & 19.5\% & 18.1\% & 17.6\% & 16.4 \% & 17.5\% \\
            \hline
            60 & 12.8\% & 18.7\% & 18.1\% & 12.6 \% & 15.1\% \\
            \hline
            100 & 16.0\% & 11.8\% & 7.8\% & 11.9 \% & 11.5\% \\
            \hline
            200 & TB & TB & 10.2\% & 13.6\% & 16.3\% \\
            \hline
        \end{tabular}

    \end{center}
    \begin{center}
    \scalebox{0.65}{
        \csvreader[%
        respect all,%
        autotabular%
      ]{data.csv}{}{\csvlinetotablerow}%
    }
    \end{center}

\begin{multicols}{2}

    \section{Conclusion}

    In conclusion the simulated annealing algorithm proved to carry an increased effectiveness in finding the optimal path of TSP problems over greedy approaches.  The findings listed above showed an 10-13\% reduction in the total cost of a TSP path over a greedy algorithm, and although a greedy approach is favorable in runtime, additional research could greatly increase a simulated annealing’s path finding optimization.  The runtime analysis proved that the difficulty in finding an effective path through simulated annealing is due in part to the randomness inherent to the algorithm itself.  ***Colin, if there’s any runtime data here to support this or any better way to incorporate the data you collected for runtime feel free to insert/alter here*** Finding valid neighboring paths turns into a guessing game with smaller odds when more cities are involved, but this risk is necessary to escape local minima and expand our search.  This finding shows that a simulated annealing approach would be much more effective at navigating local minima to find a global minimum if our city graph were fully connected and complete.   

\end{multicols}

\end{document}
